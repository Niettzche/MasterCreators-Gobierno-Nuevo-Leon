\documentclass[11pt, letterpaper]{article}
\usepackage[utf8]{inputenc}
\usepackage[spanish]{babel}
\usepackage[margin=2.5cm]{geometry}
\usepackage{xcolor}
\usepackage{helvet}
\usepackage{fancyhdr}
\usepackage{enumitem}
\usepackage{tikz}
\usetikzlibrary{shapes.geometric, arrows}

\renewcommand{\familydefault}{\sfdefault}

% Colores Institucionales
\definecolor{navyblue}{RGB}{0, 51, 102}
\definecolor{accentorange}{RGB}{255, 107, 0}
\definecolor{lightgray}{RGB}{245, 245, 245}

% Configuración de encabezado
\pagestyle{fancy}
\fancyhf{}
\rhead{\textcolor{gray}{Documento de Diseño de Experiencia (UX)}}
\lhead{\textcolor{navyblue}{\textbf{Flujos de Usuario -- Ventanilla Digital NL}}}
\rfoot{Página \thepage}

\begin{document}

% PORTADA
\begin{titlepage}
    \centering
    \vspace*{2cm}
    {\Huge \textbf{\textcolor{navyblue}{Mapeo de Flujos de Usuario (User Journeys)}}}\\[1cm]
    {\Large \textbf{Plataforma Integral de Clústeres}}\\[0.5cm]
    {\large Directorio + Ventanilla Digital de Apoyos}\\[2cm]
    
    \textbf{Versión del Documento:}\
    1.0 (Basado en Requerimientos Fase 1 MVP)\\[3cm]
    
    \textbf{Objetivo:}\
    Definir la interacción paso a paso de los distintos actores dentro del sistema para garantizar la trazabilidad y usabilidad del proceso de asignación de recursos.\\[3cm]
    
    \vfill
    \begin{flushright}
        \small 
        \textbf{Roles Principales involucrados:}\
        Clúster Solicitante | Gobierno (Admin) | Comité | Finanzas
    \end{flushright}
\end{titlepage}

% CONTENIDO
\section{Visión General de la Interacción}
El sistema se comporta como un **Gestor de Flujos de Trabajo (Workflow Engine)**. La interacción no es lineal, sino basada en estados. A continuación, se desglosan los 4 flujos críticos que componen la experiencia completa de la plataforma.

\section{Flujo 1: Onboarding y Perfilamiento (Clúster)}
\textit{Objetivo: Que un Clúster tenga una identidad digital verificada antes de solicitar dinero.}

\begin{enumerate}[label=\textbf{\arabic*.}]
    \item \textbf{Registro Inicial:} 
    El usuario (Representante del Clúster) accede a la landing page pública y selecciona "Registro de Clúster". Ingresa credenciales básicas (Correo, Contraseña) y recibe un email de verificación.
    
    \item \textbf{Configuración del Perfil (Primer Acceso):}
    Al iniciar sesión por primera vez, el sistema bloquea la navegación y fuerza al usuario a la vista de "Completar Perfil".
    \begin{itemize}
        \item \textbf{Carga de Datos:} Llena formularios de Datos Generales, Órganos de Gobierno y Comités.
        \item \textbf{Expediente Legal Base:} Sube PDFs obligatorios (Acta Constitutiva, Poder Legal, RFC).
    \end{itemize}
    
    \item \textbf{Validación (Backend/Gobierno):}
    El perfil queda en estado \textit{"En Revisión"}. Un administrador de Gobierno valida que el Clúster exista legalmente.
    
    \item \textbf{Activación:}
    Una vez validado, el Clúster recibe acceso al módulo de "Ventanilla Digital" y su información pública aparece en el "Directorio de Clústeres".
\end{enumerate}

\section{Flujo 2: La Solicitud de Apoyo (El "Wizard")}
\textit{Objetivo: Guiar al Clúster en la estructuración de un proyecto viable sin errores financieros.}

Este es el flujo principal de la UI \texttt{RegisterView/Wizard}.

\begin{enumerate}[label=\textbf{Paso \arabic*:}, leftmargin=2cm]
    \item \textbf{Selección de Convocatoria:} 
    Desde su Dashboard, el usuario ve tarjetas de "Convocatorias Abiertas". Da clic en "Aplicar".
    
    \item \textbf{Pre-llenado Automático:} 
    El sistema clona los datos del Perfil (Razón Social, RFC, Representante) en la solicitud. El usuario no los escribe, solo los confirma.
    
    \item \textbf{Definición del Proyecto:} 
    El usuario ingresa Título, Resumen, Fechas y selecciona el Rubro (Capacitación, Eventos, etc.).
    
    \item \textbf{Matriz Presupuestal (Punto Crítico):}
    \begin{itemize}
        \item El usuario agrega partidas una por una (Concepto + Costo).
        \item \textbf{Feedback en Tiempo Real:} El sistema muestra una barra de progreso de "Aportación Privada". Si es menor al 15\%, el botón "Siguiente" se deshabilita y muestra una alerta en rojo.
    \end{itemize}
    
    \item \textbf{Compromiso de Indicadores:}
    Define metas numéricas (ej. "20 Empresas Beneficiadas").
    
    \item \textbf{Firma y Envío:}
    Carga documentos específicos del proyecto (Cotizaciones). Al dar clic en "Enviar", la solicitud se bloquea para edición y cambia a estado \texttt{ENVIADA}.
\end{enumerate}

\section{Flujo 3: Evaluación y Dictamen (Gobierno)}
\textit{Objetivo: Validar técnica y administrativamente la solicitud.}

\begin{enumerate}[label=\textbf{\arabic*.}]
    \item \textbf{Filtro Administrativo (Dirección Clústeres):}
    El funcionario ve una "Bandeja de Entrada". Abre un expediente y ve la vista de "Validación Documental".
    \begin{itemize}
        \item \textit{Caso A (Todo bien):} Marca cada documento como "Válido". El estado cambia a \texttt{VALIDADA}.
        \item \textit{Caso B (Error):} Escribe un comentario en el documento erróneo (ej. "RFC ilegible") y solicita corrección. El estado regresa a \texttt{REQUIERE ACLARACIÓN}.
    \end{itemize}
    
    \item \textbf{Evaluación Técnica (Comité):}
    Los evaluadores acceden (modo lectura) a las solicitudes \texttt{VALIDADAS}.
    \begin{itemize}
        \item Asignan puntaje (0-100) según la rúbrica.
        \item Emiten voto: "Aprobado" o "Rechazado".
    \end{itemize}
    
    \item \textbf{Dictamen Final:}
    Si es aprobado, el sistema genera una notificación de "Proyecto Dictaminado Positivo" y habilita la carga del Convenio de Colaboración para firma.
\end{enumerate}

\section{Flujo 4: Ejecución Financiera y Cierre}
\textit{Objetivo: Dispersar recursos y comprobar gastos.}

\begin{enumerate}[label=\textbf{\arabic*.}]
    \item \textbf{Registro de Cuenta (Clúster):}
    El usuario sube la carátula bancaria de la cuenta exclusiva. 
    
    \item \textbf{Dispersión (Finanzas):}
    El usuario de Finanzas valida la cuenta y registra la fecha y monto de la transferencia SPEI. El estado cambia a \texttt{EN EJECUCIÓN}.
    
    \item \textbf{Comprobación de Gastos (Clúster):}
    Durante los meses del proyecto, el Clúster entra a la sección "Mis Gastos".
    \begin{itemize}
        \item Sube XML/PDF de facturas.
        \item Asocia cada factura a una "Partida Presupuestal" aprobada en el Flujo 2.
        \item El sistema valida que la suma de facturas no exceda el monto de la partida.
    \end{itemize}
    
    \item \textbf{Cierre y Auditoría:}
    Al finalizar, el Clúster sube el "Reporte Final" y evidencias (Fotos). Auditoría revisa que (Facturas + Evidencias) coincidan con lo prometido. Se cierra el proyecto.
\end{enumerate}

\vspace{1cm}

\begin{center}
    \textbf{\textit{Diagrama de Estados Simplificado (Ciclo de Vida del Proyecto)}}
    
    \vspace{0.5cm}
    
    \begin{tikzpicture}[node distance=2.5cm]
    \tikzstyle{state} = [rectangle, rounded corners, minimum width=2.5cm, minimum height=1cm,text centered, draw=black, fill=lightgray]
    \tikzstyle{arrow} = [thick,->,>=stealth]

    \node (borrador) [state] {Borrador};
    \node (enviada) [state, right of=borrador, xshift=1cm] {Enviada};
    \node (validacion) [state, right of=enviada, xshift=1cm] {Validación};
    \node (dictamen) [state, below of=validacion] {Dictamen};
    \node (ejecucion) [state, left of=dictamen, xshift=-1cm] {Ejecución};
    \node (cierre) [state, left of=ejecucion, xshift=-1cm] {Cierre};

    \draw [arrow] (borrador) -- (enviada);
    \draw [arrow] (enviada) -- (validacion);
    \draw [arrow] (validacion) -- (dictamen);
    \draw [arrow] (dictamen) -- (ejecucion);
    \draw [arrow] (ejecucion) -- (cierre);
    \draw [arrow] (validacion) to [out=135,in=45] (borrador) node[midway, above, scale=0.7] {Corrección};
    \end{tikzpicture}
\end{center}

\end{document}
