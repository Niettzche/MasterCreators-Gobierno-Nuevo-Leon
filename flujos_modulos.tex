\documentclass[11pt, letterpaper]{article}
\usepackage[utf8]{inputenc}
\usepackage[spanish]{babel}
\usepackage[margin=2.5cm]{geometry}
\usepackage{xcolor}
\usepackage{helvet}
\usepackage{fancyhdr}
\usepackage{enumitem}
\usepackage{tikz}
\usetikzlibrary{shapes.geometric, arrows, positioning}

\renewcommand{\familydefault}{\sfdefault}

% Colores Institucionales
\definecolor{navyblue}{RGB}{0, 51, 102}
\definecolor{accentorange}{RGB}{255, 107, 0}
\definecolor{lightgray}{RGB}{245, 245, 245}

% Configuración de encabezado
\pagestyle{fancy}
\fancyhf{}
\rhead{\textcolor{gray}{Flujos Detallados por Módulo}}
\lhead{\textcolor{navyblue}{\textbf{Plataforma Integral de Clústeres NL}}}
\rfoot{Página \thepage}

\begin{document}

% PORTADA
\begin{titlepage}
    \centering
    \vspace*{2cm}
    {\Huge \textbf{\textcolor{navyblue}{Diagramas de Flujo por Módulo}}}\Показать больше
    {\Large \textbf{Desglose Operativo: Directorio y Ventanilla Digital}}\Показать больше
    
    \textbf{Alcance:}\Показать больше
    Módulo A (Directorio) + Módulo B (Ventanilla Digital)\Показать больше
    \textbf{Fecha:}\Показать больше
    5 de Diciembre, 2025\Показать больше
    
    \vfill
    \begin{flushright}
        \small 
        \textbf{Documento Técnico de Referencia}\Показать больше
        Para implementación de Lógica de Negocio (Backend)
    \end{flushright}
\end{titlepage}

% CONTENIDO
\section{Introducción}
Este documento detalla la lógica secuencial de los dos grandes componentes del sistema. Cada paso representa una interacción usuario-sistema que debe ser validada por el Backend.

\newpage

% --- MODULO A ---
\section{Módulo A: Directorio de Clústeres (Registro y Validación)}
\textit{Objetivo: Crear una base de datos confiable de los 15 Clústeres estratégicos.}

\subsection*{Flujo A.1: Alta de Nuevo Clúster (Onboarding)}
\begin{enumerate}[label=\textbf{\arabic*.Показатель}]
    \item \textbf{Solicitud de Acceso:} El Representante Legal llena el formulario público "Solicitud de Registro" (Nombre, RFC, Email Oficial).
    \item \textbf{Verificación de Correo:} El sistema envía un token al email. El usuario da clic para activar la cuenta.
    \item \textbf{Primer Acceso (Modo Restringido):} El usuario entra al sistema, pero solo ve la pantalla "Completar Perfil".
    \item \textbf{Carga de Expediente Base:}
        \begin{itemize}
            \item Datos Generales (Logo, Dirección, Misión).
            \item Órganos de Gobierno (Presidente, Secretario, Comités).
            \item \textbf{Documentos Legales (Obligatorios):} Acta Constitutiva (PDF), Poder Legal (PDF), RFC (PDF).
        \end{itemize}
    \item \textbf{Validación Manual (Gobierno):}
        \begin{itemize}
            \item El Admin recibe notificación "Nuevo Clúster Registrado".
            \item Revisa que el Acta Constitutiva coincida con la Razón Social.
            \item \textit{Acción:} \textbf{Aprobar} (Habilita Módulo B) o \textbf{Rechazar} (Pide corrección).
        \end{itemize}
    \item \textbf{Publicación:} El perfil del Clúster se hace visible en el "Directorio Público" para consulta ciudadana.
\end{enumerate}

\subsection*{Flujo A.2: Registro de Empresas Asociadas}
\textit{El Clúster es responsable de mantener actualizado su padrón de socios.}
\begin{itemize}
    \item El Clúster carga su lista de socios (Nombre, RFC, Sector, Tamaño).
    \item El sistema genera estadísticas automáticas (ej. "Total de PyMEs en el sector Automotriz").
\end{itemize}

\vspace{1cm}
\begin{center}
    \textbf{\textit{Diagrama de Estados: Perfil del Clúster}}
    
    \vspace{0.5cm}
    \begin{tikzpicture}[node distance=3cm, auto]
        \node[draw, rectangle, rounded corners] (registro) {Registro Inicial};
        \node[draw, rectangle, rounded corners, right of=registro] (revision) {En Revisión (Gobierno)};
        \node[draw, rectangle, rounded corners, right of=revision] (activo) {Activo / Verificado};
        \node[draw, rectangle, rounded corners, below of=revision] (correccion) {Requiere Cambios};

        \draw[->, thick] (registro) -- (revision);
        \draw[->, thick] (revision) -- node {Aprobar} (activo);
        \draw[->, thick] (revision) -- node {Observación} (correccion);
        \draw[->, thick] (correccion) -- node {Corregir} (revision);
    \end{tikzpicture}
\end{center}

\newpage

% --- MODULO B ---
\section{Módulo B: Ventanilla Digital de Apoyos (El "Core" del Negocio)}
\textit{Objetivo: Gestionar dinero público con trazabilidad total.}

\subsection*{Flujo B.1: Configuración de Convocatoria (Gobierno)}
\begin{enumerate}[label=\textbf{\arabic*.Показатель}]
    \item Admin crea "Nueva Convocatoria 2025".
    \item Define reglas:
        \begin{itemize}
            \item Rubros permitidos: "Capacitación" y "Certificación" (Bloquea "Eventos").
            \item Tope máximo por proyecto: \$2,000,000 MXN.
            \item Fecha límite: 30 de Diciembre.
        \end{itemize}
    \item Publica la convocatoria. Aparece en el Dashboard de todos los Clústeres Activos.
\end{enumerate}

\subsection*{Flujo B.2: Solicitud de Apoyo (Clúster)}
\textit{Ver UI "Wizard" implementada.}
\begin{enumerate}[label=\textbf{\arabic*.Показатель}]
    \item \textbf{Inicio:} Clúster selecciona convocatoria y crea "Borrador".
    \item \textbf{Captura Técnica:} Define objetivos, cronograma y beneficiarios.
    \item \textbf{Captura Financiera (Validación Automática):}
        \begin{itemize}
            \item Ingresa partidas presupuestales.
            \item El sistema valida: \textit{¿Aportación Privada $\ge$ 15\%?} \textit{¿Total $\le$ Tope Máximo?}
            \item Si falla, bloquea el envío.
        \end{itemize}
    \item \textbf{Envío:} Firma digitalmente (Check de "Protesta de decir verdad"). El estado cambia a \texttt{ENVIADA}.
\end{enumerate}

\subsection*{Flujo B.3: Proceso de Aprobación (Workflow de 3 Niveles)}
\begin{enumerate}[label=\textbf{Nivel \arabic*:}]
    \item \textbf{Validación Documental (Analista):} Revisa que los PDFs sean legibles y vigentes. (Pasa a \texttt{VALIDADA} o \texttt{CORRECCIÓN}).
    \item \textbf{Evaluación Técnica (Comité):} Expertos califican el proyecto. (Pasa a \texttt{DICTAMINADA}).
    \item \textbf{Formalización (Jurídico):} Generación y firma de Convenio. (Pasa a \texttt{VIGENTE}).
\end{enumerate}

\subsection*{Flujo B.4: Ejecución y Comprobación (Finanzas)}
\begin{enumerate}[label=\textbf{\arabic*.Показатель}]
    \item \textbf{Ministración:} Gobierno registra el pago SPEI al Clúster.
    \item \textbf{Comprobación Mensual:}
        \begin{itemize}
            \item Clúster sube Factura (XML) + Evidencia (Foto del curso).
            \item Asocia el gasto a la partida "Renta de Salón".
            \item Sistema descuenta del saldo disponible de esa partida.
        \end{itemize}
    \item \textbf{Cierre:} Cuando Saldo = \$0 y Metas = 100\%, el proyecto pasa a \texttt{CERRADO / AUDITADO}.
\end{enumerate}

\vspace{1cm}
\begin{center}
    \textbf{\textit{Diagrama de Estados: Ciclo de Vida del Proyecto}}
    
    \vspace{0.5cm}
    \begin{tikzpicture}[node distance=2.2cm, auto, font=\footnotesize]
        % Nodos
        \node[draw, rectangle, rounded corners] (borrador) {1. Borrador};
        \node[draw, rectangle, rounded corners, right of=borrador, node distance=3cm] (enviada) {2. Enviada};
        \node[draw, rectangle, rounded corners, right of=enviada, node distance=3cm] (validacion) {3. Validación Doc.};
        \node[draw, rectangle, rounded corners, below of=validacion] (evaluacion) {4. Evaluación Comité};
        \node[draw, rectangle, rounded corners, left of=evaluacion, node distance=3cm] (convenio) {5. Firma Convenio};
        \node[draw, rectangle, rounded corners, left of=convenio, node distance=3cm] (ejecucion) {6. Ejecución (Pago)};
        \node[draw, rectangle, rounded corners, below of=ejecucion] (cierre) {7. Cierre / Auditoría};

        % Flechas
        \draw[->] (borrador) -- (enviada);
        \draw[->] (enviada) -- (validacion);
        \draw[->] (validacion) -- node[right] {Ok} (evaluacion);
        \draw[->] (validacion) to[bend right] node[above] {Error} (borrador);
        \draw[->] (evaluacion) -- node[above] {Aprobado} (convenio);
        \draw[->] (convenio) -- (ejecucion);
        \draw[->] (ejecucion) -- (cierre);
    \end{tikzpicture}
\end{center}

\newpage
\section{Anexo Técnico: Documentación de Dashboards}
A continuación se describe la funcionalidad detallada de los Tableros de Control (Dashboards) implementados para cada rol de usuario en la plataforma.

\subsection{1. Dashboard de Clúster (Solicitante)}
\textbf{Archivo Fuente:} \texttt{src/components/ClusterDashboard.jsx} \\
\textbf{Rol:} Clúster / Representante Legal

Este tablero funge como el centro de operaciones para los usuarios externos. Consolida la información de los dos módulos principales (Directorio y Ventanilla).

\begin{itemize}
    \item \textbf{Encabezado Principal:} Muestra la identidad del Clúster logueado y su estatus operativo (Activo/Inactivo).
    \item \textbf{Tarjeta Módulo A (Directorio):}
    \begin{itemize}
        \item Visualiza el porcentaje de completitud del perfil institucional.
        \item Muestra el contador de empresas asociadas registradas.
        \item Botones de acción: "Ver Empresas" (Listado) y "Actualizar Perfil" (Edición de datos).
    \end{itemize}
    \item \textbf{Tarjeta Módulo B (Ventanilla Digital):}
    \begin{itemize}
        \item Resumen de actividad: Solicitudes en proceso, historial y monto total aprobado.
        \item \textbf{Acción Principal:} Botón "Nueva Solicitud de Apoyo" que inicia el Wizard de captura.
    \end{itemize}
    \item \textbf{Tabla de Seguimiento:} Listado de los últimos proyectos con indicadores visuales de estatus (Borrador, En Revisión, Aprobado) y montos solicitados.
\end{itemize}

\subsection{2. Dashboard de Gobierno (Super Admin)}
\textbf{Archivo Fuente:} \texttt{src/components/SuperAdminDashboard.jsx} \\
\textbf{Rol:} Dirección de Clústeres / Admin Gobierno

Tablero de mando para la visión global del programa. Se enfoca en alertas y gestión de alto nivel.

\begin{itemize}
    \item \textbf{Grid de KPIs (Indicadores Clave):}
    \begin{itemize}
        \item \textbf{Por Validar:} Conteo de solicitudes pendientes, destacando las que requieren atención urgente.
        \item \textbf{Convocatorias Activas:} Número de convocatorias abiertas y alertas de cierre próximo.
        \item \textbf{Presupuesto Global:} Monitor del ejercicio fiscal 2025 y porcentaje de asignación.
        \item \textbf{Clústeres:} Avance de la meta de registro de los 15 clústeres estratégicos.
    \end{itemize}
    \item \textbf{Bandeja de Entrada (Feed de Actividad):} Lista cronológica de eventos del sistema (nuevas solicitudes, reportes mensuales, registros) con acciones rápidas contextuales (ej. "Revisar Documentos").
    \item \textbf{Estado de Convocatorias:} Barras de progreso que comparan el número de solicitudes recibidas vs la meta establecida para cada programa.
\end{itemize}

\subsection{3. Dashboard de Evaluación (Comité)}
\textbf{Archivo Fuente:} \texttt{src/components/EvaluatorDashboard.jsx} \\
\textbf{Rol:} Evaluador Técnico / Comité

Interfaz simplificada y focalizada en la revisión cualitativa de proyectos.

\begin{itemize}
    \item \textbf{Gestión de Asignaciones:} Muestra únicamente los expedientes asignados al usuario evaluador logueado.
    \item \textbf{Tabla de Evaluación:}
    \begin{itemize}
        \item Datos clave: Proyecto, Clúster solicitante y Fechas.
        \item \textbf{Semáforo de Fechas:} Resalta en rojo las fechas límite próximas a vencer.
        \item \textbf{Acción "Evaluar":} Redirige a la vista de detalle para capturar la calificación técnica.
    \end{itemize}
\end{itemize}

\subsection{4. Dashboard Financiero (Tesorería)}
\textbf{Archivo Fuente:} \texttt{src/components/FinanceDashboard.jsx} \\
\textbf{Rol:} Secretaría de Finanzas / Pagos

Herramienta para el control de flujos de efectivo y validación de requisitos fiscales.

\begin{itemize}
    \item \textbf{Validación de Cuentas:} KPI y acceso directo a las cuentas bancarias nuevas que requieren validación de carátula antes de poder recibir pagos.
    \item \textbf{Control de Dispersión:}
    \begin{itemize}
        \item KPI de monto total "Pendiente de Pago".
        \item Tabla detallada de proyectos aprobados listos para dispersión SPEI.
        \item Botón "Registrar Pago" para confirmar la transacción y notificar al Clúster.
    \end{itemize}
    \item \textbf{Monitor Presupuestal:} Indicador de suficiencia presupuestal en tiempo real.
\end{itemize}

\subsection{5. Dashboard de Auditoría (Contraloría)}
\textbf{Archivo Fuente:} \texttt{src/components/AuditDashboard.jsx} \\
\textbf{Rol:} Órgano Interno de Control / Auditor Externo

Enfocado en la revisión ex-post y la trazabilidad de expedientes cerrados.

\begin{itemize}
    \item \textbf{Matriz de Riesgo:} Tabla de proyectos cerrados clasificados por nivel de riesgo (Alto/Medio/Bajo) para priorizar auditorías.
    \item \textbf{Buscador Universal:} Herramienta para localizar cualquier expediente por folio, RFC o palabra clave.
    \item \textbf{Control de Observaciones:} KPI que monitorea las observaciones de auditoría abiertas y críticas pendientes de solventar.
\end{itemize}

\end{document}